\documentclass[conference]{IEEEtran}
\usepackage[colorlinks,allcolors=blue]{hyperref}
\usepackage{amsfonts}
\usepackage{graphicx}


\begin{document}

    \title{Modeling and Observing ProtoDigital Twin with ROS2 and Omniverse}
    \author{\IEEEauthorblockN{Alican Tüzün\\ and Montgomery Scott} 
    \IEEEauthorblockA{Starfleet Academy\\ San Francisco, California 966782391\\ Telephone: (800) 555--1212\\ Fax: (888) 555--1212}}
    
    \maketitle

    \begin{abstract}
        Digital twins are becoming more and more important for the efficient and effective development and operation of cyber-physical systems.
        However, digital twins are only useful if they reflect the real-world system accurately enough, i.e.their quality is high enough. 
        This claim entails the question, of what the term quality in the context of digital twins means and how it can be measured. 
        In this article, we present our experience with the quality assurance of a digital twin for an assembly line in the automotive industry.
        We explain our preliminary definition of digital twin quality, which we derive from classical quality models for general software systems. 
        Furthermore, we describe quality issues, which we were able to detect in a digital twin of an assembly line in the automotive industry. \
        Finally, we conclude how to leverage our experience in different contexts and how to generalize the underlying approaches.
    \end{abstract}

    \section{Introduction}\label{section:introduction}
    As new and complex technologies advance to emerge, it becomes progressively difficult to understand and define their core concepts. 
    One of the significant current examples in the area of digitalization is the concept of digital twins, which has resulted in numerous definitions,
    that vary from one derivation to another. Because, even though the terms "digital" and "twin" are easy to grasp,  their consolidation develops a new challenging concept,
    which might be resulted in misunderstanding and misapplication.
    Therefore, a practical example of a proto, easy-to-model example of the digital twin prototype and instance can significantly improve 
    the understanding of the notion of the digital twin.\cite{QualityandCompetitionanEssayinEconomicTheory}
    
    The idea of the digital twin was proposed in 2002 by the Grieves,
    as an ideal form of product life cycle management.
    The proposal was to create a co-existed information twin of a real system to gather new information to minimize the system needs and waste,
    such as material, time, and energy. It should be noted that initially this concept was called the mirrored spaced model, and later the information mirroring model(IMM).
    IMM had all the components of today's digital twins: a real system, a virtual system,  a connection between them, and additionally the virtual simulation. 
    However, after co-authoring with Vickers in 2010, the term 'digital twin' was adopted and Grieves simplified the model by excluding the virtual simulation component.
    
    There has been a considerable amount of literature has been published on Digital Twins, 
    however, to date, even though Grieves already gave the definition, there has been little agreement on the precise definition and application. 
    Numerous authors have considered different definitions, including digital twins as a multi-physics environment 1, an equivalent to a product 2, 
    a digital copy 3, a cyber component of a Cyber-Physical System 4, a controlling and monitoring unit 5, and many more 6. However, if inspected carefully, 
    most of these not only don't coincide with Grieves's definitions but also give some extra aspects to it. 

    Not only in the literature but digital twins also gained popularity nearly in all branches of industry,
    especially in the manufacturing scene.  This resulted in the development of general platforms to construct digital twins. 
    Furthermore, game engines such as Unity and  Unreal Engine have become popular tools for being an environment for digital twins, 
    and some of them already offer dedicated digital twin platforms. For databases, there are ontology-based cloud solutions even with newly defined 
    languages for managing the data associated with digital twins. However, while there are some accessible open-source projects, most of the digital 
    twin platforms today are not free to use and may not be easy to access or grasp. 

    To address the challenges associated with the accessibility and comprehensibility of the notion of digital twins,
    the main purpose of this paper is to follow a lifecycle model to present the lifecycle stages of an RS, with the DTI and DTP,
    but also to give observation results, including the challenges and solutions encountered during the process of creating a digital twin system.
    To create a digital twin, the author used Omniverse as a digital twin environment (DTE), and ROS2 (Robot Operating System) as a sensor and communication middleware.
    Additionally, an HC-SR04 ultrasonic sensor, Raspberry Pi, basic electronic components, and several third-party libraries have been used to construct the digital twins. 

    \section{Method}\label{section:components}
    Initially, a digital prototype representing the possible form of the real system, 
    also called a selected prototype, was developed. Second, physical components of the real system 
    relative to the digital twin prototype(DTP) were assembled. Subsequently, a connection between the real system(RS) 
    and the digital twin instance(DTI) was established, 
    resulted in the capturing of the information about the product throughout the remaining lifecycle stages.

    
    \bibliography{main}
    \bibliographystyle{plain}
    
\end{document}